\documentclass{ppgccufmg}

\usepackage[brazil]{babel}        % se o documento for em português, OU
%\usepackage[english]{babel}      % se o documento for em inglês
%\usepackage[latin1]{inputenc}
\usepackage{natbib}

\title{Title}
\date{\today}
\author{Me}

\usepackage{lipsum}

\begin{document}
	\ppgccufmg
	
	\chapter{Introdução}
		Exemplo de nota de rodapé \footnote{Lorem ipsum dolor sit amet, consectetur adipiscing elit, sed do eiusmod tempor incididunt ut labore et dolore magna aliqua.}.
		\lipsum[1-4]
		
		\section{Motivação}
			\lipsum[4-5]
			   
			\subsection{Sub-motivação}
				\lipsum[6-7]
			\subsection{Mais uma sub-motivação}
				\lipsum[8-9]
				
	\chapter{Desenvolvimento}
		\lipsum[1-5]

		\section{Usando referências}
			Segundo \cite{horn86robot}, todo triângulo equilátero tem os lados iguais. Já segundo \cite{shashua97photometric}, todo quadrado também tem.
			
			Veja que o pacote \verb|natbib| permite uma série de formas diferentes para fazer referências bibliográficas. O comando padrão, \verb|\cite|, realiza a citação comum vista no parágrafo anterior. Outros comandos permitem, por exemplo, citar somente o autor --- por exemplo, citar o trabalho de	\citeauthor{samaras99coupled} --- ou colocar automaticamente a citação entre	parênteses \citep{hougen93estimation, sato99illumination2, sato99illumination1, sato01stability}.
			
			Os comandos usados foram, respectivamente, \verb|\citeauthor| e \verb|\citep|. Veja a documentação do \verb|natbib| na Internet para conhecer	outros comandos e exemplos de uso.
			
			Citações aleatórias para fazer com que as referências bibliográficas ocupem	mais de uma página: \cite{bichsel92simple, dror01statistics, guisser92new, dwork2006calibrating, sweeney2002k}.
			
		%% Referências
		\bibliographystyle{plain}
		\bibliography{referencias}
		
		\begin{apendices}
			\chapter{Um apêndice}
				\lipsum[1-3]
				
			\chapter{Outro Apêndice}
				\lipsum[4-6]
				
		\end{apendices}
		
\end{document}
